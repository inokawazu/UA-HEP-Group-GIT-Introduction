\documentclass{article}

\usepackage{amsmath}  % This package allows the use of a large range of mathematical formula, commands, and symbols
\usepackage{graphicx}  % This package allows the importing of images
\usepackage[dvipsnames]{xcolor} %For chaning text color.
\usepackage{hyperref}%For linking to URLs



\title{How to Access Git Via Your Desktop Computer} %The presentation title.
%\date{} %The date, if important
\author{Markus A. Garbiso \\ The University of Alabama} %Me and the place I work.


%For highlighing important
\newcommand{\lf}[1]{\textcolor{RoyalBlue}{{\bf Needs Further Research:} #1}}

%For highlighing important
\newcommand{\cn}[1]{\textcolor{WildStrawberry}{{\bf Citation Needed:} #1}}

%Typing GitHub is hard.
\newcommand{\gh}{GitHub~}



\begin{document}

\maketitle

\section{Accessing Git}

Marco covered how to use \gh with the terminal, but you can also use \gh's repository features with \textbf{\gh Desktop} (\href{https://desktop.github.com/}{https://desktop.github.com/}). Once you have downloaded the program and installed it you should be able to access all the features of the terminal and a nice GUI. Once you have a \gh account you can access all of our private repositories. By default you will not be able to see any repositories. The first thing you want to do is to ``Clone repository''.  You can select one of your own repositories from a list or any other repository you have access to via it's URL. Now you will be able to select a branch! I will go over some of the \gh lingo. You have several tabs.

\begin{itemize}
  \item File
  \begin{itemize}
  	\item \textit{New repository: } You can add a new repository to publish onto \gh.
  	\item \textit{Clone repository: } Use this for repositories that are shared and on \gh.
  \end{itemize}
  \item Edit\\This is mostly for the cases were you make a mistake and wanted to undo some fatal error.
  \item Repository\\This the where you will make contributions to the repository, so this menu will see most of the use.\footnote{Note the shortcuts which can save a siginificant amount of time.}
   \begin{itemize}
  	\item \textit{Push: } Used to push current uncommited commits onto \gh.
  	\item \textit{Pull: } Used to pull/Fetch from the repository on \gh.
  	\item \textit{Remove...: } this will delete the repository on the local machine (and on \gh).
  	\item \textit{View on Github: } This will open you browser to \gh.
  \end{itemize}
  \item Branch\\This is used to make new branches and other commands like to remove all recent changes from that branch.
\end{itemize}

\section{Making a Contribution}

\subsection{Commits}

\subsection{Branches}

\subsection{Pull Request}

\subsection{Merging Branches}

\end{document}