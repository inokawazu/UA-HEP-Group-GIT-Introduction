\documentclass{article}

\usepackage{amsmath}  % This package allows the use of a large range of mathematical formula, commands, and symbols
\usepackage{graphicx}  % This package allows the importing of images
\usepackage[dvipsnames]{xcolor} %For chaning text color.
\usepackage{hyperref}%For linking to URLs



\title{How to Access Git Via Your Desktop Computer} %The presentation title.
%\date{} %The date, if important
\author{Markus A. Garbiso \\ The University of Alabama} %Me and the place I work.


%For highlighing important
\newcommand{\lf}[1]{\textcolor{RoyalBlue}{{\bf Needs Further Research:} #1}}

%For highlighing important
\newcommand{\cn}[1]{\textcolor{WildStrawberry}{{\bf Citation Needed:} #1}}





\begin{document}

\maketitle

\section{Accessing Git}

Marco covered how to use GitHub with the terminal, but you can also use Github's repository features with \textbf{Git Hub Desktop} (\href{https://desktop.github.com/}{https://desktop.github.com/}). Once you have downloaded the program and installed it you should be able to access all the features of the terminal and a nice GUI. Once you have a GitHub account

\section{Making a Contribution}

\section{Commits}

\section{Branches}

\section{Pull Request}

\section{Merging Branches}

\end{document}