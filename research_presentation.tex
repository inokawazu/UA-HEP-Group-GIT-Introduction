\documentclass{article}

\usepackage{amsmath}  % This package allows the use of a large range of mathematical formula, commands, and symbols
\usepackage{graphicx}  % This package allows the importing of images
\usepackage[dvipsnames]{xcolor} %For chaning text color.
\usepackage{hyperref}%For linking to URLs



\title{How to Access Git Via Your Desktop Computer} %The presentation title.
%\date{} %The date, if important
\author{Markus A. Garbiso \\ The University of Alabama} %Me and the place I work.


%For highlighing important
\newcommand{\lf}[1]{\textcolor{RoyalBlue}{{\bf Needs Further Research:} #1}}

%For highlighing important
\newcommand{\cn}[1]{\textcolor{WildStrawberry}{{\bf Citation Needed:} #1}}

%Typing GitHub is hard.
\newcommand{\gh}{GitHub~}



\begin{document}

\maketitle

\section{Accessing Git}

Marco covered how to use \gh with the terminal, but you can also use \gh's repository features with \textbf{\gh Desktop} (\href{https://desktop.github.com/}{https://desktop.github.com/}). Once you have downloaded the program and installed it you should be able to access all the features of the terminal and a nice GUI. Once you have a \gh account you can access all of our private repositories. By default you will not be able to see any repositories. The first thing you want to do is to ``Clone repository''.  You can select one of your own repositories from a list or any other repository you have access to via it's URL. Now you will be able to select a branch! I will go over some of the \gh lingo. You have several tabs.

\begin{itemize}
  \item File
  \begin{itemize}
  	\item \textit{New repository: } You can add a new repository to publish onto \gh.
  	\item \textit{Clone repository: } Use this for repositories that are shared and on \gh.
  \end{itemize}
  \item Edit\\This is mostly for the cases were you make a mistake and wanted to undo some fatal error.
  \item Repository\\This the where you will make contributions to the repository, so this menu will see most of the use.\footnote{Note the shortcuts which can save a siginificant amount of time.}
   \begin{itemize}
  	\item \textit{Push: } Used to push current uncommited commits onto \gh.
  	\item \textit{Pull: } Used to pull/Fetch from the repository on \gh.
  	\item \textit{Remove...: } this will delete the repository on the local machine (and on \gh).
  	\item \textit{View on Github: } This will open you browser to \gh.
  \end{itemize}
  \item Branch\\This is used to make new branches and other commands like to remove all recent changes from that branch.
\end{itemize}

\section{Making a Contribution}

\gh shines when used to collaborate on code development. I will got through the \textit{workflow} of making a contribution.

\subsection{Branches}

The directory of \gh is the branch. By default every repository has a \textit{master branch}. Usually if you are in collaboration, you would rarely make minor changes to this branch. When in \gh Desktop you can create a new branch via the \textit{Branch}. Be sure you are on the correct \textit{Current Branch} before you start to make changes (Commits). You will also have to Sync by \textit{Fetching} the data from the \gh version. You will be able to see a button to fetch.

\subsection{Commits}

The unit of change in \gh are the \textit{commits}. Once you make a change to \textbf{any} file, your \gh Desktop will show you a list of changes along with the list of files. If \gh Desktop can read the file (like a code file) it will attempt to show the changes. Once you have given your commits a name (\textit{Summary}), you can then upload the changes via a \textit{push}.

\textit{The Git Philosophy: ``...Being able to work offline is yet another aspect which differentiates it from other tools out there. Just think of the advantage of not having to go over the internet for each and every operation! This makes diff’s, commits etc seamless and fast. The only time you need to have the git server available is when you decide to push your code or make your code available to others.''}

It helps to keep commits \textbf{small} and such that any commits don't leave the code ``broken''.

\subsection{Pull Request}

Once you have complete the commits enough on a branch, you can then perform a \textit{Pull Request} where the changes on the branch you made commits to will be merged into the master branch\footnote{Our whatever you branch you branched out of}. I reccomend you request someone on the \gh site so that you have a second opinion before the branches are merged.

\subsection{Merging Branches}

When merging branches, sometimes you will have merge conflicts. This is when \gh doesn't know how to merge the two versions of a file. You will have to then go in manually and tell \gh which \textit{branch} to keep.

\section{Conclusion}

This is the end of using \gh Desktop though I would like to mention that \gh online offers sever features to orginze workflow. I will be using \textit{Projects} and \textit{Issues}, so it's clear where the group is at on a certain code for a project.

\end{document}